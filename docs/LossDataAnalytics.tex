\documentclass[]{book}
\usepackage{lmodern}
\usepackage{amssymb,amsmath}
\usepackage{ifxetex,ifluatex}
\usepackage{fixltx2e} % provides \textsubscript
\ifnum 0\ifxetex 1\fi\ifluatex 1\fi=0 % if pdftex
  \usepackage[T1]{fontenc}
  \usepackage[utf8]{inputenc}
\else % if luatex or xelatex
  \ifxetex
    \usepackage{mathspec}
  \else
    \usepackage{fontspec}
  \fi
  \defaultfontfeatures{Ligatures=TeX,Scale=MatchLowercase}
\fi
% use upquote if available, for straight quotes in verbatim environments
\IfFileExists{upquote.sty}{\usepackage{upquote}}{}
% use microtype if available
\IfFileExists{microtype.sty}{%
\usepackage{microtype}
\UseMicrotypeSet[protrusion]{basicmath} % disable protrusion for tt fonts
}{}
\usepackage[margin=1in]{geometry}
\usepackage{hyperref}
\hypersetup{unicode=true,
            pdftitle={Loss Data Analytics},
            pdfauthor={An open text authored by the Actuarial Community},
            pdfborder={0 0 0},
            breaklinks=true}
\urlstyle{same}  % don't use monospace font for urls
\usepackage{natbib}
\bibliographystyle{econPeriod}
\usepackage{longtable,booktabs}
\usepackage{graphicx,grffile}
\makeatletter
\def\maxwidth{\ifdim\Gin@nat@width>\linewidth\linewidth\else\Gin@nat@width\fi}
\def\maxheight{\ifdim\Gin@nat@height>\textheight\textheight\else\Gin@nat@height\fi}
\makeatother
% Scale images if necessary, so that they will not overflow the page
% margins by default, and it is still possible to overwrite the defaults
% using explicit options in \includegraphics[width, height, ...]{}
\setkeys{Gin}{width=\maxwidth,height=\maxheight,keepaspectratio}
\IfFileExists{parskip.sty}{%
\usepackage{parskip}
}{% else
\setlength{\parindent}{0pt}
\setlength{\parskip}{6pt plus 2pt minus 1pt}
}
\setlength{\emergencystretch}{3em}  % prevent overfull lines
\providecommand{\tightlist}{%
  \setlength{\itemsep}{0pt}\setlength{\parskip}{0pt}}
\setcounter{secnumdepth}{5}
% Redefines (sub)paragraphs to behave more like sections
\ifx\paragraph\undefined\else
\let\oldparagraph\paragraph
\renewcommand{\paragraph}[1]{\oldparagraph{#1}\mbox{}}
\fi
\ifx\subparagraph\undefined\else
\let\oldsubparagraph\subparagraph
\renewcommand{\subparagraph}[1]{\oldsubparagraph{#1}\mbox{}}
\fi

%%% Use protect on footnotes to avoid problems with footnotes in titles
\let\rmarkdownfootnote\footnote%
\def\footnote{\protect\rmarkdownfootnote}

%%% Change title format to be more compact
\usepackage{titling}

% Create subtitle command for use in maketitle
\providecommand{\subtitle}[1]{
  \posttitle{
    \begin{center}\large#1\end{center}
    }
}

\setlength{\droptitle}{-2em}

  \title{Loss Data Analytics}
    \pretitle{\vspace{\droptitle}\centering\huge}
  \posttitle{\par}
    \author{An open text authored by the Actuarial Community}
    \preauthor{\centering\large\emph}
  \postauthor{\par}
    \date{}
    \predate{}\postdate{}
  
\usepackage{booktabs}
\setcounter{secnumdepth}{2}

\begin{document}
\maketitle

{
\setcounter{tocdepth}{2}
\tableofcontents
}
\chapter*{Preface}\label{preface}
\addcontentsline{toc}{chapter}{Preface}

\emph{Date: 20 April 2019}

\subsubsection*{Book Description}\label{book-description}
\addcontentsline{toc}{subsubsection}{Book Description}

\textbf{Loss Data Analytics} is an interactive, online, freely available
text.

\begin{itemize}
\tightlist
\item
  The online version contains many interactive objects (quizzes,
  computer demonstrations, interactive graphs, video, and the like) to
  promote \emph{deeper learning}.
\item
  A subset of the book is available for \emph{offline reading} in pdf
  and EPUB formats.
\item
  The online text will be available in multiple languages to promote
  access to a \emph{worldwide audience}.
\end{itemize}

\subsubsection*{What will success look
like?}\label{what-will-success-look-like}
\addcontentsline{toc}{subsubsection}{What will success look like?}

The online text will be freely available to a worldwide audience. The
online version will contain many interactive objects (quizzes, computer
demonstrations, interactive graphs, video, and the like) to promote
deeper learning. Moreover, a subset of the book will be available in pdf
format for low-cost printing. The online text will be available in
multiple languages to promote access to a worldwide audience.

\subsubsection*{How will the text be
used?}\label{how-will-the-text-be-used}
\addcontentsline{toc}{subsubsection}{How will the text be used?}

This book will be useful in actuarial curricula worldwide. It will cover
the loss data learning objectives of the major actuarial organizations.
Thus, it will be suitable for classroom use at universities as well as
for use by independent learners seeking to pass professional actuarial
examinations. Moreover, the text will also be useful for the continuing
professional development of actuaries and other professionals in
insurance and related financial risk management industries.

\subsubsection*{Why is this good for the
profession?}\label{why-is-this-good-for-the-profession}
\addcontentsline{toc}{subsubsection}{Why is this good for the
profession?}

An online text is a type of open educational resource (OER). One
important benefit of an OER is that it equalizes access to knowledge,
thus permitting a broader community to learn about the actuarial
profession. Moreover, it has the capacity to engage viewers through
active learning that deepens the learning process, producing analysts
more capable of solid actuarial work.

Why is this good for students and teachers and others involved in the
learning process? Cost is often cited as an important factor for
students and teachers in textbook selection (see a recent post on the
\href{https://www.aei.org/publication/the-new-era-of-the-400-college-textbook-which-is-part-of-the-unsustainable-higher-education-bubble/}{\$400
textbook}). Students will also appreciate the ability to ``carry the
book around'' on their mobile devices.

\subsubsection*{Why loss data analytics?}\label{why-loss-data-analytics}
\addcontentsline{toc}{subsubsection}{Why loss data analytics?}

The intent is that this type of resource will eventually permeate
throughout the actuarial curriculum. Given the dramatic changes in the
way that actuaries treat data, loss data seems like a natural place to
start. The idea behind the name \emph{loss data analytics} is to
integrate classical loss data models from applied probability with
modern analytic tools. In particular, we recognize that big data
(including social media and usage based insurance) are here to stay and
that high speed computation is readily available.

\subsubsection*{Project Goal}\label{project-goal}
\addcontentsline{toc}{subsubsection}{Project Goal}

The project goal is to have the actuarial community author our textbooks
in a collaborative fashion. To get involved, please visit our
\href{https://sites.google.com/a/wisc.edu/loss-data-analytics/}{Open
Actuarial Textbooks Project Site}.

\section*{Acknowledgements}\label{acknowledgements}
\addcontentsline{toc}{section}{Acknowledgements}

Edward Frees acknowledges the John and Anne Oros Distinguished Chair for
Inspired Learning in Business which provided seed money to support the
project. Frees and his Wisconsin colleagues also acknowledge a Society
of Actuaries Center of Excellence Grant that provided funding to support
work in dependence modeling and health initiatives.

We acknowledge the Society of Actuaries for permission to use problems
from their examinations.

We thank Rob Hyndman, Monash University, for allowing us to use his
excellent style files to produce the online version of the book.

We thank Yihui Xie and his colleagues at
\href{https://www.rstudio.com/}{Rstudio} for the
\href{https://bookdown.org/yihui/bookdown/}{R bookdown} package that
allows us to produce this book.

We also wish to acknowledge the support and sponsorship of the
\href{http://www.blackactuaries.org/}{International Association of Black
Actuaries} in our joint efforts to provide actuarial educational content
to all.

\includegraphics[width=0.25000\textwidth]{Figures/IABA.png}

\section*{Contributors}\label{contributors}
\addcontentsline{toc}{section}{Contributors}

The project goal is to have the actuarial community author our textbooks
in a collaborative fashion. The following contributors have taken a
leadership role in developing \emph{Loss Data Analytics}.

\begin{itemize}
\item
  \textbf{Zeinab Amin} is the Director of the Actuarial Science Program
  and Associate Dean for Undergraduate Studies of the School of Sciences
  and Engineering at the American University in Cairo (AUC). Amin holds
  a PhD in Statistics and is an Associate of the Society of Actuaries.
  Amin is the recipient of the 2016 Excellence in Academic Service Award
  and the 2009 Excellence in Teaching Award from AUC. Amin has designed
  and taught a variety of statistics and actuarial science courses.
  Amin's current area of research includes quantitative risk assessment,
  reliability assessment, general statistical modelling, and Bayesian
  statistics.
\item
  \textbf{Katrien Antonio}, KU Leuven
\item
  \textbf{Jan Beirlant}, KU Leuven
\item
  \textbf{Arthur Charpentier}, Université du Quebec á Montreal
\end{itemize}

\begin{itemize}
\tightlist
\item
  \textbf{Curtis Gary Dean} is the Lincoln Financial Distinguished
  Professor of Actuarial Science at Ball State University. He is a
  Fellow of the Casualty Actuarial Society and a CFA charterholder. He
  has extensive practical experience as an actuary at American States
  Insurance, SAFECO, and Travelers. He has served the CAS and actuarial
  profession as chair of the Examination Committee, first
  editor-in-chief for \emph{Variance: Advancing the Science of Risk},
  and as a member of the Board of Directors and the Executive Council.
  He contributed a chapter to \emph{Predictive Modeling Applications in
  Actuarial Science} published by Cambridge University Press.
\end{itemize}

\begin{itemize}
\tightlist
\item
  \textbf{Edward W. (Jed) Frees} is an emeritus professor, formerly the
  Hickman-Larson Chair of Actuarial Science at the University of
  Wisconsin-Madison. He is a Fellow of both the Society of Actuaries and
  the American Statistical Association. He has published extensively (a
  four-time winner of the Halmstad and Prize for best paper published in
  the actuarial literature) and has written three books. He also is a
  co-editor of the two-volume series \emph{Predictive Modeling
  Applications in Actuarial Science} published by Cambridge University
  Press.
\end{itemize}

\begin{itemize}
\item
  \textbf{Guojun Gan} is an assistant professor in the Department of
  Mathematics at the University of Connecticut, where he has been since
  August 2014. Prior to that, he worked at a large life insurance
  company in Toronto, Canada for six years. He received a BS degree from
  Jilin University, Changchun, China, in 2001 and MS and PhD degrees
  from York University, Toronto, Canada, in 2003 and 2007, respectively.
  His research interests include data mining and actuarial science. He
  has published several books and papers on a variety of topics,
  including data clustering, variable annuity, mathematical finance,
  applied statistics, and VBA programming.
\item
  \textbf{Lisa Gao} is a doctoral student at the University of
  Wisconsin-Madison.
\item
  \textbf{José Garrido}, Concordia University
\end{itemize}

\begin{itemize}
\tightlist
\item
  \textbf{Lei (Larry) Hua} is an Associate Professor of Actuarial
  Science at Northern Illinois University. He earned a PhD degree in
  Statistics from the University of British Columbia. He is an Associate
  of the Society of Actuaries. His research work focuses on multivariate
  dependence modeling for non-Gaussian phenomena and innovative
  applications for financial and insurance industries.
\end{itemize}

\begin{itemize}
\tightlist
\item
  \textbf{Noriszura Ismail} is a Professor and Head of Actuarial Science
  Program, Universiti Kebangsaan Malaysia (UKM). She specializes in Risk
  Modelling and Applied Statistics. She obtained her BSc and MSc
  (Actuarial Science) in 1991 and 1993 from University of Iowa, and her
  PhD (Statistics) in 2007 from UKM. She also passed several papers from
  Society of Actuaries in 1994. She has received several research grants
  from Ministry of Higher Education Malaysia (MOHE) and UKM, totaling
  about MYR1.8 million. She has successfully supervised and
  co-supervised several PhD students (13 completed and 11 on-going). She
  currently has about 180 publications, consisting of 88 journals and 95
  proceedings.
\end{itemize}

\begin{itemize}
\item
  \textbf{Joseph H.T. Kim}, Ph.D., FSA, CERA, is Associate Professor of
  Applied Statistics at Yonsei University, Seoul, Korea. He holds a
  Ph.D.~degree in Actuarial Science from the University of Waterloo, at
  which he taught as Assistant Professor. He also worked in the life
  insurance industry. He has published papers in \emph{Insurance
  Mathematics and Economics}, \emph{Journal of Risk and Insurance},
  \emph{Journal of Banking and Finance}, \emph{ASTIN Bulletin}, and
  \emph{North American Actuarial Journal}, among others.
\item
  \textbf{Shyamalkumar Nariankadu} - University of Iowa
\end{itemize}

\begin{itemize}
\item
  \textbf{Nii-Armah Okine} is a dissertator at the business school of
  University of Wisconsin-Madison with a major in actuarial science. He
  obtained his master's degree in Actuarial science from Illinois State
  University. His research interests includes micro-level reserving,
  joint longitudinal-survival modeling, dependence modelling, micro
  insurance and machine learning.
\item
  \textbf{Margie Rosenberg} - University of Wisconsin
\end{itemize}

\begin{itemize}
\item
  \textbf{Emine Selin Sarıdaş} is a doctoral candidate in the Statistics
  department of Mimar Sinan University. She holds a bachelor degree in
  Actuarial Science with a minor in Economics and a master degree in
  Actuarial Science from Hacettepe University. Her research interest
  includes dependence modeling, regression, loss models and life
  contingencies.
\item
  \textbf{Peng Shi} - University of Wisconsin - Madison
\item
  \textbf{Jianxi Su}, Purdue University
\item
  \textbf{Tim Verdonck}, KU Leuven
\end{itemize}

\begin{itemize}
\tightlist
\item
  \textbf{Krupa Viswanathan} is an Associate Professor in the Risk,
  Insurance and Healthcare Management Department in the Fox School of
  Business, Temple University. She is an Associate of the Society of
  Actuaries. She teaches courses in Actuarial Science and Risk
  Management at the undergraduate and graduate levels. Her research
  interests include corporate governance of insurance companies, capital
  management, and sentiment analysis. She received her Ph.D.~from The
  Wharton School of the University of Pennsylvania.
\end{itemize}

\section*{Reviewers}\label{reviewers}
\addcontentsline{toc}{section}{Reviewers}

Our goal is to have the actuarial community author our textbooks in a
collaborative fashion. Part of the writing process involves many
reviewers who generously donated their time to help make this book
better. They are:

\begin{itemize}
\tightlist
\item
  Yair Babab
\item
  Chunsheng Ban, Ohio State University
\item
  Vytaras Brazauskas, University of Wisconsin - Milwaukee
\item
  Chun Yong Chew, Universiti Tunku Abdul Rahman (UTAR)
\item
  Eren Dodd, University of Southampton
\item
  Gordon Enderle, University of Wisconsin - Madison
\item
  Rob Erhardt, Wake Forest University
\item
  Runhun Feng, University of Illinois
\item
  Liang (Jason) Hong, Robert Morris University
\item
  Fei Huang, Australian National University
\item
  Hirokazu (Iwahiro) Iwasawa
\item
  Himchan Jeong, University of Connecticut
\item
  Min Ji, Towson University
\item
  Paul Herbert Johnson, University of Wisconsin - Madison
\item
  Samuel Kolins, Lebonan Valley College
\item
  Andrew Kwon-Nakamura, Zurich North America
\item
  Ambrose Lo, University of Iowa
\item
  Mark Maxwell, University of Texas at Austin
\item
  Tatjana Miljkovic, Miami University
\item
  Bell Ouelega, American University in Cairo
\item
  Zhiyu (Frank) Quan, University of Connecticut
\item
  Jiandong Ren, Western University
\item
  Rajesh V. Sahasrabuddhe, Oliver Wyman
\item
  Ranee Thiagarajah, Illinois State University
\item
  Ping Wang, Saint Johns University
\item
  Chengguo Weng, University of Waterloo
\item
  Toby White, Drake University
\item
  Michelle Xia, Northern Illinois University
\item
  Di (Cindy) Xu, University of Nebraska - Lincoln
\item
  Lina Xu, Columbia University
\item
  Lu Yang, University of Amersterdam
\item
  Jorge Yslas, University of Copenhagen
\item
  Jeffrey Zheng, Temple University
\item
  Hongjuan Zhou, Arizona State University
\end{itemize}

\section*{For our Readers}\label{for-our-readers}
\addcontentsline{toc}{section}{For our Readers}

We hope that you find this book worthwhile and even enjoyable. For your
convenience, at our \href{https://openacttexts.github.io/}{Github
Landing site} (\url{https://openacttexts.github.io/}), you will find
links to the book that you can (freely) download for offline reading,
including a pdf version (for Adobe Acrobat) and an EPUB version suitable
for mobile devices.
\href{https://github.com/OpenActTexts/Loss-Data-Analytics/tree/master/Data}{Data}
for running our examples are available at the same site.

In developing this book, we are emphasizing the
\href{https://openacttexts.github.io/Loss-Data-Analytics/index.html}{online
version} that has lots of great features such as a glossary, code and
solutions to examples that you can be revealed interactively. For
example, you will find that the statistical code is hidden and can only
be seen by clicking on terms such as

R Code for Frequency Table

\hypertarget{display.T:Frequency.2Intro}{}
\begin{verbatim}
Insample <- read.csv("Insample.csv", header=T,  na.strings=c("."), stringsAsFactors=FALSE)
Insample2010 <- subset(Insample, Year==2010)
table(Insample2010$Freq)
\end{verbatim}

We hide the code because we don't want to insist that you use the
\texttt{R} statistical software (although we like it). Still, we
encourage you to try some statistical code as you read the book -- we
have opted to make it easy to learn \texttt{R} as you go. We have even
set up a separate \href{https://openacttexts.github.io/LDARcode}{R Code
for Loss Data Analytics} site to explain more of the details of the
code.

Freely available, interactive textbooks represent a new venture in
actuarial education and we need your input. Although a lot of effort has
gone into the development, we expect hiccoughs. Please let your
instructor know about opportunities for improvement, write us through
the discussion features in the online text, or contact chapter
contributors directly with suggested improvements.

\section{Relevance of Analytics to Insurance Activites}\label{S:Intro}

\subsection{Nature and Relevance of
Insurance}\label{nature-and-relevance-of-insurance}

\subsection{What is Analytics?}\label{what-is-analytics}

\subsection{Insurance Processes}\label{S:InsProcesses}

\section{Insurance Company Operations}\label{S:PredModApps}

\subsection{Initiating Insurance}\label{initiating-insurance}

\subsection{Renewing Insurance}\label{renewing-insurance}

\subsection{Claims and Product
Management}\label{claims-and-product-management}

\subsection{Loss Reserving}\label{S:Reserving}

\section{Case Study: Wisconsin Property Fund}\label{S:LGPIF}

\subsection{Fund Claims Variables: Frequency and
Severity}\label{S:OutComes}

\subsection{Fund Rating Variables}\label{S:FundVariables}

\subsection{Fund Operations}\label{fund-operations}

\section{Further Resources and
Contributors}\label{Intro-further-reading-and-resources}

\chapter{Frequency Modeling}\label{C:Frequency-Modeling}

Placeholder

\section{Frequency Distributions}\label{S:frequency-distributions}

\subsection{How Frequency Augments Severity
Information}\label{S:how-frequency-augments-severity-information}

\subsubsection{Basic Terminology}\label{S:basic-terminology}

\subsubsection{The Importance of
Frequency}\label{S:the-importance-of-frequency}

\subsubsection{Why Examine Frequency
Information}\label{S:why-examine-frequency-information}

\section{Basic Frequency
Distributions}\label{S:basic-frequency-distributions}

\subsection{Foundations}\label{S:foundations}

\subsection{Moment and Probability Generating
Functions}\label{S:generating-functions}

\subsection{Important Frequency
Distributions}\label{S:important-frequency-distributions}

\subsubsection{Binomial Distribution}\label{S:binomial-distribution}

\subsubsection{Poisson Distribution}\label{S:poisson-distribution}

\subsubsection{Negative Binomial
Distribution}\label{S:negative-binomial-distribution}

\section{The (a, b, 0) Class}\label{S:the-a-b-0-class}

\section{Estimating Frequency
Distributions}\label{S:estimating-frequency-distributions}

\subsection{Parameter Estimation}\label{S:parameter-estimation}

\subsection{Frequency Distributions
MLE}\label{S:frequency-distributions-mle}

\section{Other Frequency
Distributions}\label{S:other-frequency-distributions}

\subsection{Zero Truncation or
Modification}\label{S:zero-truncation-or-modification}

\section{Mixture Distributions}\label{S:mixture-distributions}

\section{Goodness of Fit}\label{S:goodness-of-fit}

\section{Exercises}\label{S:exercises}

\section{R Code for Plots in this Chapter}\label{S:rcode}

\section{Further Resources and
Contributors}\label{Freq-further-reading-and-resources}

\subsubsection*{Contributors}\label{contributors-1}
\addcontentsline{toc}{subsubsection}{Contributors}

\bibliography{Bibliography/LDAReferenceC.bib}


\end{document}
