\documentclass[serif,10pt]{beamer}
%%%%%%%%%%%%%%%%%%%%%%%%%%%%%%%%%%%%%%%%%%%%%%%%%%%%%%%%%%%%%%%%%%%%%%%%%%%%%%%%%%%%%%%%%%%%%%%%%%%%%%
\mode<presentation> {
  \usetheme[left,width=1.5cm]{UWThemeC}
%    \usetheme[left,hideallsubsections,width=1cm]{UWThemeB}
  \usefonttheme[onlymath]{serif}
}

\usepackage{alltt}
\usepackage{tikz}                   % For TikZ graphicss
\usepgflibrary{patterns}            % This is to get "dots" and stuff in the graphics
\usepgflibrary{snakes}              % This is to get the "snakes" in the LD picture
\usepackage{hyperref}
\usepackage[english]{babel}
\usepackage[latin1]{inputenc}
\usepackage{mathptmx}               % Hace lucir la matematica rara
\usepackage{helvet}
\usepackage{courier}

\usepackage{natbib}
%\bibliographystyle{asa}
\bibliographystyle{econPeriod}

\usepackage[absolute,overlay]{textpos}
\usepackage{graphicx}
\usepackage{multirow}
\usepackage{scalefnt}
\usepackage[T1]{fontenc}            % Note that the encoding and the font should match. If T1
                                    % does not look nice, try deleting the line with the fontenc.
\usepackage{colortbl}
\usepackage{color}
\definecolor{light-gray}{gray}{0.75}
\definecolor{blizzardblue}{rgb}{0.67, 0.9, 0.93}
\definecolor{bleudefrance}{rgb}{0.19, 0.55, 0.91}
\def\textcol{blue}

\usepackage{hyperref}
\hypersetup{
    bookmarks=true,         % show bookmarks bar?
    unicode=false,          % non-Latin characters in Acrobat's bookmarks
    pdftoolbar=true,        % show Acrobat's toolbar?
    pdfmenubar=true,        % show Acrobat's menu?
    pdffitwindow=true,      % page fit to window when opened
    pdftitle={My title},    % title
    pdfauthor={Author},     % author
    pdfsubject={Subject},   % subject of the document
    pdfnewwindow=true,      % links in new window
    pdfkeywords={keywords}, % list of keywords
    colorlinks=true,        % false: boxed links; true: colored links
    linkcolor=black,         % color of internal links
    citecolor=bleudefrance,        % color of links to bibliography
    filecolor=magenta,      % color of file links
    urlcolor=cyan           % color of external links
}

\setlength\fboxsep{3pt}
\setlength\fboxrule{1pt}

\setbeamertemplate{blocks}[rounded][shadow=true]
\setbeamertemplate{navigation symbols}{}

%Personal definitions - Bayes Regression
\def\bsb{\boldsymbol \beta}
\def\bsa{\boldsymbol \alpha}
\def\bsm{\boldsymbol \mu}
\def\bsS{\boldsymbol \Sigma}
\def\bsx{\boldsymbol \xi}

\title[Predictive Analytics and Medical Errors]{Predictive Analytics and Medical Errors \\
~\\
{\small{\textit{Advances in Predictive Analytics} \\
~ \\
University of Waterloo}}
}

\author[Frees]{Edward W. (Jed) Frees }

\institute [Univ Wisconsin] {
 Joint work with Lisa Gao \\
  ~~\\
  University of Wisconsin -- Madison    }

\date[December 2017]{December 2017}


\begin{document}

\frame{\titlepage}

\begin{frame}
  \frametitle{Outline}
    %\tableofcontents[part=1,pausesections]
     \tableofcontents[part=1]
\end{frame}

\begin{frame}
\frametitle{Research Team}
\begin{figure}[htp]
%\begin{center}
%\hspace*{-.5in}
\includegraphics[width=0.25\textwidth]{FiguresMPL/Frees_Jed_150x188.jpg}
~~~~~~~~~~\includegraphics[width=0.25\textwidth]{FiguresMPL/Gao_Lisa_150x188.jpg}
\end{figure}
\end{frame}

\part<presentation>{Main Talk}



\section[Introduction]{Resources for Introductions to Insurance Analytics}

\begin{frame}\label{AnalyticsOverview}
\frametitle{Insurance Analytics Resources}

In this first part of the talk, I intend to introduce resources
  \begin{itemize}
  \item for actuaries wishing to learn more about analytics
  \begin{itemize}
\item  A two volume series, published by \textit{Cambridge University Press}.
\end{itemize}
\item for statisticians/machine learners/financial engineers wishing to learn more about insurance company operations
\begin{itemize}\item A review paper entitled \href{http://www.annualreviews.org/doi/abs/10.1146/annurev-financial-111914-041815}{\textit{Analytics of Insurance Markets}}, in the \href{http://www.annualreviews.org/journal/financial}{\textit{Annual Review of Financial Economics} 2016}.
\end{itemize}

\end{itemize}
\end{frame}

\subsection{Book Series}


\begin{frame}
%\frametitle{Predictive Modeling Book}
\vspace{-.8in}
\begin{figure}[htp]
    \includegraphics[width=1.05\textwidth]{Figures/PredModelBookWeb.png}
\end{figure}
\end{frame}

\begin{frame}
\frametitle{Predictive Modeling Book}
  \begin{itemize}
  \item We coordinated and co-authored this two volume set, published by Cambridge University Press, that provides the foundations of statistical modeling for actuaries interested in learning about predictive analytics
  \begin{itemize}  \item I am a co-Editor, along with Glenn Meyers and Richard Derrig
  \item Authors from 7 different countries
  \end{itemize}
\item Book URL \url{http://research.bus.wisc.edu/PredModelActuaries}
\item Volume 2, on case studies, appeared last year
\item Translated into Japanese by the Japenese Institute of Actuaries
\item Co-sponsored by the Canadian Institute of Actuaries
\end{itemize}
\end{frame}


\subsection[Overview Paper]{Insurance Company Analytics}

\begin{frame}
\frametitle{What is Analytics?}

A review paper entitled \href{http://www.annualreviews.org/doi/abs/10.1146/annurev-financial-111914-041815}{\textit{Analytics of Insurance Markets}}, in the \href{http://www.annualreviews.org/journal/financial}{\textit{Annual Review of Financial Economics}}.

 \begin{itemize}
 \item Insurance is a data-driven industry -- analytics is a key to deriving information from data.
  \item But what is analytics? \pause \ \ \ \ Some alternative descriptors:
   \begin{itemize}
 \item ``business intelligence'' may focus on processes of collecting data, often through databases and data warehouses
 \item ``business analytics'' utilizes tools and methods for statistical analyses of data
  \item ``data science'' can encompass broader applications in many scientific domains
  \end{itemize}
  \pause
  \item \textcolor{red}{Analytics} -- the process of using data to make decisions.
     \begin{itemize}
  \item This process involves gathering data, understanding models of uncertainty, making general inferences, and communicating results.
    \end{itemize}
\end{itemize}\end{frame}



\begin{frame}
\frametitle{What is Analytics?}
 \begin{itemize}
  \item Led by statistician W. Edwards Deming, an earlier generation sought to utilize quality improvement techniques to improve business processes, resulting in the field now known as ``total quality management.'' \pause
  \item Analytics continues to enjoy increasing popularity among businesses.
\end{itemize}
\begin{figure}[htp]
\begin{center}
\includegraphics[width=.8\textwidth]{Figures/HBROverhead.png}
  \end{center}
\end{figure}
\end{frame}



\begin{frame}
\frametitle{Statistics and Predictive Analytics for Insurance}
Why ``Predictive''? \pause
\vspace{-.05in}
  \begin{itemize}
 \item Statisticians think about the traditional triad of inference: hypothesis testing, parameter estimation, and prediction.
 \item In insurance, predictions are useful for existing risks in future periods as well as not yet observed risks in a current period
 \end{itemize}
\vspace{-.1in}
   \begin{figure}[htp]
\begin{center}
       \caption{\label{F:PredAnalytics}\small Predictive Features of Insurance Analytics, Norberg (1979).}
\vspace{-.2in}
    \includegraphics[width=0.65\textwidth]{Figures/NorbergExhibit8A.PNG}
\end{center}
\end{figure}
\end{frame}


\section{Medical Errors and Medical Malpractice}

\begin{frame}%[shrink=20]
\frametitle{Adverse Events and Medical Errors}
  \begin{itemize}
\item \textbf{Adverse events} are defined as unintended injuries or complications caused by health care management, rather than by the patient's underlying disease, that lead to death, disability or prolonged hospital stays.
 \begin{itemize}\item From the influential \textcolor{bleudefrance}{\textit{The Canadian Adverse Events Study}, 2006}.
\item An example is \textit{nosocomial infection}, an infection acquired during hospital care that are not present or incubating at admission. These infections lead to approximately 10,000 deaths per year, making this the fourth leading cause of death in Canada. \end{itemize}
\item \textbf{Medical error} is an unintended act or one that does not achieve its intended outcome, an error of execution, an error of planning, or a deviation from the process of care that may or may not cause harm to the patient.
 \begin{itemize}\item From \textcolor{bleudefrance}{Makary and Daniel (2016, \textit{British Medical Journal})} where medical error is labeled as the \textcolor{blue}{third leading cause of death} in the US\end{itemize}

\end{itemize}
\end{frame}

\begin{frame}%[shrink=20]
\frametitle{Medical Malpractice}
  \begin{itemize}
\item Not all adverse events, nor even all medical errors, qualify as instances of medical malpractice.
\begin{itemize}\item There can be no malpractice without an established practice and the healthcare provider is \textit{negligent} in failing to exercise due care.
\item There are standards of fault that must be met in determining whether negligence has occurred
\item In the US, this is through the tort system where the plaintiff must establish the four elements of a lawsuit: (1) duty, (2) breach, (3) proximate cause, and (4) damages.
\end{itemize}
\item Malpractice is just one component of medical errors but represents a large amount of spending
\begin{itemize}\item   Some estimate malpractice spending to be on the order of \$20-50 billion USD annually.
\item Malpractice concerns give rise to \textit{defensive medicine}, defined as the ordering of tests, referrals, and other services primarily, though not solely, to reduce liability risk; or avoidance of high-risk services or patients.
\end{itemize}
\end{itemize}
\end{frame}

\subsection{Historical Perspective}

\begin{frame}
\frametitle{How did Medical Malpractice Suits Start in the US?}
From \cite{mohr2000american}, the \textit{Journal of the American Medical Association}
  \begin{itemize}
\item Legal framework available from English principles even before the birth of the nation, e.g., Blackstone, \textit{Commentaries on the Laws of England} in 1768.
\item First lawsuits appeared in the 1840's and well established in the 1850's.
\item Lawsuits arose in part due to changes in attitudes about role of health and medicines as well as the role of professions in policing their membership.
  \begin{itemize}
\item For example, some prominent physicians encouraged legal action against those practicing medicine who were not well-trained nor well-educated.
\item \textcolor{blue}{Even the most egregious Quacks escape punishment as things now stand}
\end{itemize}
\end{itemize}
~~~~~~~~~~~~prominent physician Nathan Smith, Yale University, 1827.
\end{frame}

\begin{frame}
\frametitle{Why do Suits Continue? - Medical Reasons}
  \begin{itemize}
\item \textcolor{blue}{Medical progress}. The competitive nature of the US medical marketplace is about innovation. In prior generations, healers used practices of their predecessors. The push to introduce new devices and new techniques means the introduction of new risks. \cite{mohr2000american} gives the example of the introduction of x-ray technology at the beginning of the 1900's. Radiographs became a source of malpractice actions (too much radiation, failure to read films properly, and so forth).
\item  \textcolor{blue}{There can be no malpractice without an established practice.} The American Medical Association was founded in 1847 to regularize education and establish uniform national standards.
\item Physicians pioneered the introduction of \textcolor{blue}{liability insurance} at the end of the 1800's. With insurance, now every physician was worth suing, not just the wealthy ones.
\end{itemize}
\end{frame}

\begin{frame}
\frametitle{Why do Suits Continue? - Legal Reasons}
  \begin{itemize}
\item \textcolor{blue}{Contingent fee arrangements}. As early as the 1880's, Mohr states that "the AMA regarded a large proportion of malpractice actions as having "no other foundation that a desire to extort money from the defendant sufficient to secure a good fee for the prosecuting counsel."
\item \textcolor{blue}{Jury trials}. Even as early as the 1850's, prominent physicians campaigned for specially empaneled expert juries, something rarely done in the US legal system.
\item \textcolor{blue}{Tort versus contracts}. {\small {In the early part of the 1800's, malpractice was treated as a tort, "a private civil wrong that violate understood duties or social responsibilities." Suits in the second half of the 1800's leaned toward malpractice as a contract, an agreement where both parties have theoretically agreed in advance on on acceptable and unacceptable outcomes. However, contracts are viewed as agreements between equals, something that the physicians themselves did not like (they didn't want to be classified like boilermakers or other nonprofessional associations). As a tort, malpractice is vague, flexible, and easy to manipulate.}}
\end{itemize}
\end{frame}

\begin{frame}
\frametitle{ Chandler Case}
  \begin{itemize}
\item To emphasize the growth of malpractice awards beginning in the 1960's, one source (\cite{yang2016lumbar}) emphasizes the Chandler case.
\item Jeff Chandler was a well-known movie star in the 1950's, he stood 6' 4" and was labeled as ``impossibly handsome.''
\end{itemize}
\begin{figure}[htp]
\begin{center}
    \includegraphics[width=4cm]{FiguresMPL/JeffChandler.PNG}
    %\caption{\label{F:ElasticNet}\small Leis Equations}
    \end{center}
\end{figure}
\end{frame}

\begin{frame}
\frametitle{Chandler Case - Continued}
  \begin{itemize}
\item He died in 1961 from complications due to a rupture in blood lining from a ``routine'' back surgery (aortic-iliac injury in a lumbar diskectomy procedure).
\item The case was settled for \$235K plus expenses. This award dwarfed other cases at the time.
 \begin{itemize}\item He was well-known by the public (27 box-office successful movies) and at the prime of his career (42 years old at death).\end{itemize}
\item The case was settled by Culver City Hospital as the defendant.
 \begin{itemize}\item Just before, a 1957 case altered the practice of practice of charitable immunity, switching the prime target of malpractice litigation from the physician to the more affluent hospitals.
 \item Charitable immunity held that a charitable or nonprofit organization could not be held liable under tort law.
 \end{itemize}
\end{itemize}
\end{frame}

\subsection{International Perspective}

\begin{frame}
\frametitle{Goals of a Medical Malpractice System}
Medical malpractice systems exist world-wide. What are their goals?
\begin{itemize}
\item \textbf{Prevention.} The prevention of medical injuries and the promotion of patient safety are paramount goals of health care policy. The prospect of liability in damages acts as an incentive to act with reasonable care.
\item \textbf{Compensation.} Compensation of injured patients is a core function of the law regarding medical malpractice and medical injuries
\begin{itemize}
\item Distributive justice - compensation for all personal injuries, howsoever caused?
\item Corrective justice - compensation for ``undue'' interference in plaintiff's rights?
\end{itemize}
\item \textbf{Accountability.} Injured patients want to know what went wrong, who was responsible for it, what efforts are being made to prevent future repetitions, and to receive an admission of fault and an apology.
\end{itemize}
\end{frame}

\begin{frame}
\frametitle{US -- Benchmark}
For better or worse, medical malpractice in the US has acted as a yardstick in global debates on the topic.

The US is different because:
  \begin{itemize}
  \item the role of the jury in deciding upon liability and assessing damages
\item the availability of punitive damages
\item the rule that each party bear their own legal costs, win or lose,
\item the large role played by contingency fees
\item the availability of extensive pre-trial procedures to require disclosure of documents and the taking of witness statements.
\end{itemize}
From \textcolor{bleudefrance}{Oliphant (2013)}.
\end{frame}

\begin{frame}[shrink=20]
\frametitle{Canada -- Less Costly Alternative}
Six factors have contained the volume and cost of malpractice litigation in Canada.
  \begin{itemize}
\item \textcolor{blue}{Non-economic damages} for personal injury were \textcolor{blue}{capped} by the Supreme Court of Canada in the late 1970s.
\item \textcolor{blue}{There is no pressure to reform from physicians}, because the dominant insurance scheme, overseen by the Canadian Medical Protective Association (CMPA), effectively insulates physicians from the impact of tort liability; a finding of medical malpractice does not drive up an individual's insurance premiums. The provinces contribute significantly to this cushioning effect, by reimbursing a significant portion of CMPA fees.
\item The CMPA has used its \textcolor{blue}{deep pockets} in responding to medical malpractice claims which has discouraged litigation.
\item Canada's rules for \textcolor{blue}{awarding costs} contribute to this problem, by making it risky for plaintiffs to pursue uncertain claims.
\item The \textcolor{blue}{inherent difficulties} in establishing causation in medical malpractice cases combined with the broad defenses available for physicians further exacerbate the uphill battle facing the plaintiff patient.
\item The tort law system treats physicians as \textcolor{blue}{independent warriors} shielding hospitals from vicarious liability for their malpractice.
\end{itemize}
From \cite{flood2011canadian}.
\end{frame}

\begin{frame}%[shrink=20]
\frametitle{New Zealand -- No Fault System}
  \begin{itemize}
\item In New Zealand, a statutory accident compensation scheme makes provision for the payment of compensation to the victims of personal injury that is suffered in ways that are covered by the scheme.
\item One of these ways is personal injury caused by medical treatment.
  \begin{itemize}
\item The statutory entitlements available to victims of personal injury are treatment and rehabilitation, earnings-related compensation, lump sum compensation for permanent impairment, and death benefits.
\item Lump sums may be awarded to compensate for permanent impairment, but not for pain and suffering.
\end{itemize}
\item From the scheme's inception, there has been a \textcolor{blue}{bar on suing} in New Zealand for damages for personal injuries or death.
\item This is a compulsory, state-controlled, scheme, funded by levies in the nature of taxation, in a field which traditionally has been the preserve of private action and initiative.
\end{itemize}
From \cite{todd2011treatment}
\end{frame}

\section{US Tort Reform}

\begin{frame}
\frametitle{Size of US Malpractice Awards}

How big do claims get???

\cite{bixenstine2014catastrophic} in {\small \textit{Catastrophic Medical Malpractice Payouts in the United States}}

  \begin{itemize}
\item Defines a \textit{catastrophic} claim to be one where the payout exceeds \$1 million USD
\item Uses NPDB (National Practitioners' Data Bank) over 2004-2010
 \begin{itemize}
\item 6,130 catastrophic payouts
\item represents 7.9\% of all 77,621 reported paid claims
\end{itemize}
\item The median catastrophic payout was \$1,148,574, maximum was \$31,744,521.
\item  The 7-year nationwide total of all catastrophic malpractice payouts was \$9.8 billion
 \begin{itemize} \item represents 36.2\% of the total payouts (\$27.0 billion).
\item Annually, about \$1.4 billion per year.
\end{itemize}
\end{itemize}
\end{frame}


\begin{frame}
\frametitle{Tort Reform in the US}
\textbf{Can we bring the US medical malpractice system under control?}

  \begin{itemize}
\item In the US, the determination of liability and compensation is almost entirely the province of the civil tort system.
\item Plaintiffs may not recover for injuries stemming from adverse events or errors unless a provider was negligent
\item Tort ``reform'' is favored by organized medicine, a perception that likely explains stakeholders persistent interest in pursuing damages caps and other conventional tort reforms.
\begin{itemize}
\item The primary objective of tort reforms is to reduce the volume and cost of malpractice litigation.
\item This is done through: i) impose barriers to bringing suits or reaching trial, ii) limit the amount of compensation that plaintiffs may recover, and iii) change how damages awards are paid.
\end{itemize}\end{itemize}
\end{frame}



\begin{frame}
\frametitle{Limitations/Caps on Damages}

 \begin{itemize}
\item A specific type of tort reform is a limitation, or cap, on damages.
\item Evidence suggests (cf., \cite{kachalia2011new}) that this is the only type of reform that has had an impact on settlements.
  \begin{itemize}
\item Limitations are placed on the monetary compensation that can be awarded in a malpractice trial for noneconomic losses, economic losses,  or both.
\item An example of a non-economic damage award is money for ``pain and suffering.''
\item A cap may apply to the plaintiff, limiting the amount that the plaintiff may receive,or to a defendant, limiting the total amount that the defendant may be required to pay.
\end{itemize}
\end{itemize}
\end{frame}



\begin{frame}%[shrink=20]
\frametitle{Medical Professional Liability Tort Reforms}
\scalefont{0.60}

%\bigskip

\begin{tabular}{l|l} \hline \hline
\textbf{Reform} &\textbf{Description }\\ \hline
\textcolor{blue}{Caps on} & Limitations are placed on the monetary compensation that can be awarded  \\
~~~\textcolor{blue}{damages} & ~~~ in a malpractice trial for noneconomic losses, economic losses, or both.  \\ \hline
Screening  & Expert panels review malpractice cases at an early stage and provide opinions \\
~~~panels &  about whether claims have sufficient merit to proceed. \\ \hline
Certificate of & The plaintiff must present, at the time of filing a malpractice claim or soon thereafter,  \\
~~~ merit & ~~~an affidavit certifying that a qualified medical expert believes \\
  &~~~ that there is reasonable and meritorious cause for the suit. \\ \hline
Limits on & Limitations are placed on the amount that a plaintiff's attorney may take as \\
~~~attorney fees&  a contingency fee. A limitation is typically expressed as a percen1tage of the \\
&~~~award, but it may also incorporate a maximum dollar value. \\\hline
Joint and & In malpractice trials involving multiple defendants, JSL reform limits the financial liability  \\
~~~  several& ~~~of each defendant to the percentage of fault that the jury allocates to that defendant. \\
~~~ liability (JSL)  &  ~~~ Without this statutory reform, a plaintiff may collect the entire judgment \\
&~~~ from one defendant, regardless of that defendant's extent of fault in the case. \\ \hline
Collateral & This reform eliminates a traditional rule that even if an injured plaintiff has received \\
~~~ source rule & ~~~ compensation from other sources (e.g., health insurance),  the amount \\
&~~~ should not be deducted from the amount that a defendant who is found liable must pay. \\ \hline
Periodic& This reform allows or requires insurers to pay malpractice awards over a long period  \\
~~~ payment  & ~~~ of time rather than in a lump sum. Insurers are also able to retain any amount that is \\
&~~~ not collected during a plaintiff's lifetime. \\ \hline
Statute of   & These statutes limit the amount of time that a patient has to file a malpractice claim  \\
~~~limitations & ~~~ after being injured or discovering an injury. \\ \hline\hline
\end{tabular}

\bigskip

\textit{Source: }\cite{kachalia2011new}
\end{frame}

\begin{frame}
\frametitle{Two Recent Court Rulings}
  \begin{itemize}
\item On July 5, 2017, the Wisconsin Appellate court recently struck down the limit that caps non-economic damages in the State of Wisconsin,
\begin{itemize}\item This case may be reviewed by the Wisconsin Supreme Court. If not, it becomes state law.\end{itemize}
\item In a similar fashion, on June 8, 2017, the Florida Supreme Court struck down the limitation on capping non-economic damages for the state of Florida,
\end{itemize}
\end{frame}


\section{Wisconsin Injured Patients Fund}

\begin{frame}
\frametitle{Wisconsin Medical Professional Liability (MPL) Market}
Wisconsin is a middle size state, located in the middle part of the US
\begin{itemize}
\item Population: 5.8 million, rank 20
\item GDP: 309 billion, rank 20
\item NAIC P\&C premiums: 10.3 billion, rank 22
\item NAIC MPL premiums: 76.4 million, rank 28
\end{itemize}
\vspace{-.1in}
\begin{figure}[htp]
    \includegraphics[width=0.6\textwidth]{FiguresMPL/USA_map_WI_location.jpg}
\end{figure}
\end{frame}


\begin{frame}
\frametitle{Patient Compensation Funds}
  \begin{itemize}
\item To help manage the Medical Professional Liability (MPL) marketplace, nine states have Patient Compensation Funds (PCFs)
  \begin{itemize}
\item Goal 1: provide physicians an excess layer of coverage and thus decreases the volatility in losses
\item Goal 2: ensure that sufficient funds are available to provide compensation for injured patients
\end{itemize}
\item You can think of these funds as acting as reinsurers for the MPL market
  \begin{itemize}
\item Every PCF state requires physicians to carry primary medical malpractice liability insurance
\item Fund participation is mandatory in only three states: Kansas, Pennsylvania, and Wisconsin
\item Wisconsin is the only state where fund liability is \textbf{unlimited}
\end{itemize}
\end{itemize}
\end{frame}

\begin{frame}
\frametitle{Patient Fund Comparison}
\scalefont{0.80}
\begin{table}[htbp]
  \centering
    \begin{tabular}{|c|rr|rr|}
    \hline
           & \multicolumn{2}{c|}{\textbf{Primary Carrier Requirements}} & \multicolumn{2}{c|}{\textbf{Fund Coverage Limits}} \\
     STATE & \multicolumn{1}{c}{Per}  & \multicolumn{1}{r|}{Annual}  & \multicolumn{1}{c}{Per}  & \multicolumn{1}{r|}{Annual}  \\
           & \multicolumn{1}{c}{Occurrence} &  \multicolumn{1}{r|}{Aggregate} & \multicolumn{1}{c}{Occurrence} &  \multicolumn{1}{r|}{Aggregate} \\
    \hline
    Indiana & 250,000  & 750,000  & 1,000,000  &  \\
    %\hline
    Kansas & 200,000  & 600,000  & 100,000  & 300,000  \\
           &       &       & 300,000  & 900,000  \\
          &       &       & 800,000*  & 2,400,000*  \\
    \hline
    Louisiana & 100,000  & 300,000  & 400,000  &  \\
    %\hline
    Nebraska & 500,000  & 1,000,000  & 1,750,000  &  \\
    \hline
    New Mexico & 200,000  & 600,000  & 400,000  &  \\
    %\hline
    New York & 1,300,000  & 3,900,000  & 1,000,000  & 3,000,000  \\
    \hline
    Pennsylvania & 500,000  & 1,500,000  & 500,000  & 1,500,000  \\
    %\hline
    South Carolina & 200,000  & 600,000  & Variable**  & Variable** \\
    \hline
    Wisconsin & 1,000,000  & 3,000,000  & \textbf{Unlimited} & \textbf{Unlimited} \\
    \hline
    \end{tabular}
  \label{tab:addlabel}
\end{table}

\bigskip
*Most physicians in Kansas opt for the largest coverage choice \\
**South Carolina is similar to Kansas but with more options
\end{frame}


\begin{frame}
\frametitle{Wisconsin Injured Patients Fund}
  \begin{itemize}\scalefont{0.9}
\item Participation is mandatory
\begin{itemize}
\item For individual physicians, Certified Registered Nurse Anesthetists (CRNA), and hospitals
\item Not available for other healthcare providers, e.g., dentists.
\end{itemize}
\item Coverage
\begin{itemize}
\item Participants are responsible for the first layer, \$1,000,000 per occurrence/\$3,000,000 annual aggregate.
\item The fund is responsible for the excess.\end{itemize}
\item Annual assessments (premiums) are about 20 million
\item Fund currently has 1.3 billion in assets
\item The state runs a company (Wisconsin Health Care Liability Insurance Plan, WHCLIP) for the residual market
\end{itemize}%\scalefont{1.1111}
\pause


\textbf{Research Question} - Suppose that Wisconsin eliminates the limitation on awards for non-economic (e.g., pain and suffering) damages. Is this a big deal?

\bigskip

What do we know about the effects of changing limitations on caps of non-economic damages on a medical malpractice system?

\end{frame}


\begin{frame}
\frametitle{Effects of Damage Caps on Claims Frequency}
  \begin{itemize}
\item Several studies have found that the adoption of liability reform lowers the probability of physicians experiencing a malpractice claim, \cite{avraham2007empirical}, \cite{danzon1986frequency}, \cite{kessler2002liability}
\item Others have found no effect \cite{danzon1984frequency}, \cite{zuckerman1990effects}, \cite{durrance2009noneconomic}
\item \cite{muhlestein2016caps} in {\small \textit{Caps on Noneconomic Damages' Effect on the Number of Paid Malpractice Claims in various American States}}
\begin{itemize}
\item Of the 15 states that implemented caps on noneconomic damages or significantly changed their caps since 2000, only two had significant changes in the number of paid claims (intercept) and six had significant changes to their trend of paid claims (slope).
\item Uses NPDB (National Practitioners' Data Bank)
\end{itemize}

\end{itemize}
\end{frame}

\begin{frame}
\frametitle{Wisconsin NPDB Number of Claims}

Wisconsin, which raised its caps, did not see a significant increase in the rate of paid claims.
\begin{figure}[htp]
\begin{center}
    \includegraphics[width=6cm]{FiguresMPL/NumberClaimsWisconsin.PNG}
    \end{center}
\end{figure}
\vspace{-.2in}
Based on NPDB data -- ignores hospital claims.

\end{frame}


\begin{frame}
\frametitle{Effects of Damage Caps on Claims Severity}
 \begin{itemize}
\item Several studies have found that limitations on allowable damage recovery, particularly for noneconomic damages, reduce the average size of malpractice payments. \cite{avraham2007empirical}, \cite{danzon1984frequency}, \cite{danzon1986frequency}, \cite{sloan1985state}, \cite{yoon2001damage}.
\item Others have found no effect. \cite{zuckerman1990effects}, \cite{durrance2009noneconomic}
\item \cite{seabury2014medical} in {\small \textit{Medical Malpractice Reform: Noneconomic Damages Caps Reduced Payments 15 Percent, With Varied Effects By Specialty,}}
\begin{itemize}
\item Found that a cap reduces average payments by 15\%, with the reduction varying by the size of the cap.
\item Study uses PIAA (Physicians Insurers Association of America) data over 1985-2010; has information on hospital claims.
\item They were able to control for the physician specialty, an advantage compared to NPDB.
\item Limitation - No information on exposure/frequency of claims.
\end{itemize}
\end{itemize}
\end{frame}

\begin{frame}
\frametitle{Other Measures of Effects of Damage Caps}
 \begin{itemize}
\item \cite{born2014differential} in {\small \textit{The Differential Effects Of Noneconomic Damage Cap Levels On Medical Malpractice Insurers,}} study the effects of caps on insurance performance.
 \begin{itemize}
 \item Here, direct losses incurred and the loss ratio are measures of performance
 \item They use NAIC data from 1997 to 2007 and consider several types of caps and of varying degrees
 \end{itemize}
 \item \cite{friedson2017medical} in {\small \textit{Medical Malpractice Damage Caps And Provider Reimbursement,}} estimate the effect of damage caps on the amount providers charge to insurance companies as
well as the amount that insurance companies reimburse providers for medical services.
 \item \cite{paik2017damage} in {\small \textit{Damage Caps and Defensive Medicine, Revisited,}} examines effects on Medicare spending for hospital care
 \item Others...
\end{itemize}

\end{frame}

\begin{frame}
\frametitle{Relating Caps on Damages to Policy Limits}

  \begin{itemize}
\item \cite{silver2008malpractice} in {\small \textit{Malpractice Payouts and Malpractice Insurance: Evidence from Texas Closed Claims, 1990-2003}}.
  \begin{itemize}
\item  Texas is the only (publicly available) source that captures information on limits on coverage of a medical professional liability insurance policy.
\item  ``As a practical matter, the stakes in malpractice suits are capped by the limits of physicians' insurance coverage, regardless of the severity of patients' injuries or the amounts that juries believe patients ought to receive.''\end{itemize}
\item Several studies have established that the imposition of a cap has some (but limited)  impact on the amounts paid. However, for Wisconsin, the situation may be more dire....
\end{itemize}
\end{frame}


\begin{frame}
\frametitle{Wisconsin History with Non-Economic Caps on MPL}
Caps on non-economic damages has varied over time within Wisconsin

\begin{table}\scalefont{0.9}
    \begin{tabular}{l|l}
    \hline
\textbf{Period} & \textbf{Change in Cap}\\    \hline
1975-1985& No caps in place when the fund began, July 1, 1975. \\ \hline
1986-1991& A \$1 million cap was introduced June 14, 1986.\\
1991-1995& Cap eliminated, beginning January 1, 1991.\\ \hline
1995-2005& \$350,000 cap adjusted upward for inflation \\
&~~~introduced May 25, 1995.\\
2005-2006& Cap eliminated, July 15, 2005.\\ \hline
2006-2017& \$750,000 cap introduced April 1, 2006.\\
2017 - present& Wisconsin District 1 Appellate Court \\
&~~~determines caps are unconstitutional (July 5, 2017)\\
    \hline
    \end{tabular}
\end{table}

Can we use this variation to detect changes in claiming patterns?
\end{frame}

\begin{frame}
\frametitle{Wisconsin Reported Claims I}
  \begin{itemize}
\item In Wisconsin, the Patients Fund acts as the reinsurer for medical professional liability cases. As such, virtually all claims are reported to the system.
\item In part because data are not regularly reviewed, the quality of the data is uneven.
 \begin{itemize}
 \item In other databases, e.g., NPDB, claims are counted as reported only when some payment is made (indemnity or expense).
\item For this database, a claim is counted as reported if the Fund is made aware of it.
\end{itemize} \pause
\item There are 13,365 claims in this database (1975-2017)
\item We split regimes into the year before a regulatory changed occurred as well as a year after.
\item When considering effects of statutory changes on insurance outcomes, customary to recognize that
\begin{itemize}
\item there may be ``lead'' effects in that plaintiffs are well aware of changes before they actually occur.
\item there may be lag effects in that some plaintiffs only appreciate the importance of regulatory effects until after they are in place.
\end{itemize}

\end{itemize}
\end{frame}

\begin{frame}
\frametitle{Wisconsin Reported Claims II}
\vspace{-.2in}
\begin{figure}[htp]
    \includegraphics[width=0.9\textwidth]{FiguresMPL/WisconsinMPLClaimsNumber.pdf}
\end{figure}
\vspace{-.3in}
There are no dramatic changes in the number of claims surrounding the two regimes where caps were eliminated, 1991-1995 and 2005-2006.
\end{frame}


\begin{frame}
\frametitle{Wisconsin Injured Patients Fund Claims}
  \begin{itemize}
\item A more disciplined method of counting claims consists of claims for which the fund made a payment.
\item There are 765 such claims that total 915.2, in millions of dollars.
\item The attachment point at which fund payments are made has varied over time.
\end{itemize}

\begin{table}\scalefont{0.9}
    \begin{tabular}{l|l}
    \hline
\textbf{Period} & \textbf{Fund Payment Obligations}\\
&\textit{Fund pays damages in excess of:}  \\ \hline
1975-1987 & \$200,000 ~~(\$600,000 aggregate) until July 1, 1987.\\
1987-1988&  \$300,000 ~~(\$900,000 aggregate) until July 1, 1988.\\
1988-1997&   \$400,000 ~~(\$1,000,000 aggregate) until July 1, 1997.\\
1997-present&   \$1,000,000 (\$3,000,000 aggregate)\\
    \hline
    \end{tabular}
\end{table}
\end{frame}


\begin{frame}
\frametitle{Patient Fund Number of Claims}
\vspace{-.2in}
\begin{figure}[htp]
    \includegraphics[width=0.9\textwidth]{FiguresMPL/PatientFundClaimsNumber.pdf}
\end{figure}
\vspace{-.3in}
There was substantial volatility following the end of the periods in which caps were eliminated in 1986 and 1995 although not 2006.
\end{frame}



\begin{frame}
\frametitle{Patient Fund Average Claim Amounts}
\vspace{-.2in}
\begin{figure}[htp]
    \includegraphics[width=0.9\textwidth]{FiguresMPL/PatientFundClaimsAmount.pdf}
\end{figure}
\vspace{-.3in}
For years surrounding the regime 2005-2006 when caps were eliminated, the typical (average) damage size is larger compared to surrounding years.
\end{frame}


\begin{frame}
\frametitle{References}
\scalefont{0.4}
\bibliography{MPLOctober2017}
\scalefont{1.25}
\end{frame}


\end{document}

\begin{frame}
\frametitle{xxx}
  \begin{itemize}
\item xxx
\end{itemize}
\end{frame}

\textcolor{blue}{temp}

